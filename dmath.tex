\documentclass[a4paper, 12pt]{article}

\usepackage{cmap}
\usepackage[T2A]{fontenc}
\usepackage[utf8]{inputenc}
\usepackage{amssymb}
\usepackage{graphicx}
\graphicspath{ {./images/} }
\usepackage[english, russian]{babel}
\usepackage{mathtools}
\usepackage[top=1in, bottom=1in, left=1.25in, right=1.25in]{geometry}
\newcommand{\parag}[1]{\paragraph{#1}\mbox{}\\}

\begin{document}
\section{Определения}

\parag{1. Логические операции: конъюнкция, дизъюнкция и отрицание.}
1) \underline{Конъюнкция} - это сложное логическое выражение, которое считается истинным в том и только том случае, когда оба простых выражения являются истинными, во всех остальных случаях данное сложеное выражение ложно.

Обозначение: F = A $\land$ B.

Таблица истинности для конъюнкции:

\begin{center}
    \begin{tabular}{|c|c|c|}
        \hline
        A & B & A $\land$ B  \\
        \hline
        0 & 0 & 0  \\
         \hline
        0 & 1 & 0  \\
        \hline
        1 & 0 & 0 \\
        \hline
        1 & 1 & 1 \\
        \hline
    \end{tabular}
\end{center}

\noindent
2) \underline{Дизъюнкция} - это сложное логическое выражение, которое истинно, если хотя бы одно из простых логических выражений истинно и ложно тогда и только тогда, когда оба простых логических выраженныя ложны.

Обозначение: F = A $\vee$ B.

Таблица истинности для дизъюнкции:

\begin{center}
    \begin{tabular}{|c|c|c|}
        \hline
        A & B & A $\vee$ B  \\
        \hline
        0 & 0 & 0  \\
         \hline
        0 & 1 & 1  \\
        \hline
        1 & 0 & 1 \\
        \hline
        1 & 1 & 1 \\
        \hline
    \end{tabular}
\end{center}

\noindent
3) \underline{Отрицание}  - это сложное логическое выражение, в котором если исходное логическое выражение истинно, то результат отрицания будет ложным, и наоборот, если исходное логическое выражение ложно, то результат отрицания будет истинным.

Обозначение: F = $\neg$A

Таблица истинности для отрицания:
\begin{center}
    \begin{tabular}{|c|c|}
        \hline
        A & $\neg$A  \\
        \hline
        0 & 1  \\
         \hline
        1 & 0  \\
        \hline
    \end{tabular}
\end{center}

\parag{2. Логические операции: импликация, XOR (исключающее или) и эквивалентность.}
1) \underline{Импликация} - это сложное логическое выражение, которое истинно во всех случаях, кроме случая, когда из истины следует ложь. То есть данная логическая операция связывает два простых логических выражения, из которых первое является условием (А), а второе (В) является следствием.

Обозначение: F = A $\to$ B

Таблица истинности для импликации:

\begin{center}
    \begin{tabular}{|c|c|c|}
        \hline
        A & B & A $\to$ B  \\
        \hline
        0 & 0 & 1  \\
         \hline
        0 & 1 & 1  \\
        \hline
        1 & 0 & 0 \\
        \hline
        1 & 1 & 1 \\
        \hline
    \end{tabular}
\end{center}

2) \underline{XOR (исключающее или)} -  это сложное логическое выражение, которое является истинным тогда и только тогда, когда оба простых логических выражения имеют разную истинность.

Обозначение: F = A $\oplus$ B

Таблица истинности для исключающего или:

\begin{center}
    \begin{tabular}{|c|c|c|}
        \hline
        A & B & A $\oplus$ B  \\
        \hline
        0 & 0 & 0  \\
         \hline
        0 & 1 & 1  \\
        \hline
        1 & 0 & 1 \\
        \hline
        1 & 1 & 0 \\
        \hline
    \end{tabular}
\end{center}


3) \underline{Эквивалентность} - это сложное логическое выражение, которое является истинным тогда и только тогда, когда оба простых логических выражения имеют одинаковую истинность.

Обозначение: F = A $\leftrightarrow$ B 

Таблица истинности для эквивалентности:

\begin{center}
    \begin{tabular}{|c|c|c|}
        \hline
        A & B & A $\leftrightarrow$ B  \\
        \hline
        0 & 0 & 1  \\
         \hline
        0 & 1 & 0  \\
        \hline
        1 & 0 & 0 \\
        \hline
        1 & 1 & 1 \\
        \hline
    \end{tabular}
\end{center}

\parag{3. Булевы функции. Задание таблицей истинности и вектором значений.}
1) Булева функция - функция от N аргументов из N-ой степени множества E = \{0,1\} в множество E = \{0,1\}. То есть:

\[
f: E^{n}_{2} \to E_{2}, E_{2} = \{0, 1\}
\]

Булеву функцию от N переменных можно задать \textit{таблицей истинности}:

\begin{center}
    \begin{tabular}{|c c c c c|c|}
        \hline
        $x_{1}$ & $x_{2}$ & ... & $x_{n-1}$ & $x_{n}$ & $f(x_{1}, x_{2}, ..., x_{n-1}, x_{n})$  \\
        \hline
        0 & 0 & ... & 0 & 0 & f(0, 0, ..., 0, 0)  \\
        0 & 0 & ... & 0 & 1 & f(0, 0, ..., 0, 1)  \\
        ... & ... & ... & ... & ... & ...  \\
        1 & 1 & ... & 1 & 0 & f(1, 1, ..., 1, 0)  \\
        1 & 1 & ... & 1 & 1 & f(1, 1, ..., 1, 1) \\
        \hline
    \end{tabular}
\end{center}

\noindent
Значения переменных можно не хранить если принять соглашение о перечислении наборов переменных в определенном порядке. Обычно таким порядком принимается порядок возрастания целых чисел, заданных наборами переменных как двоичными числами. (Еще этот порядок называют \textit{установленным}). Таблицу истинности можно "транспонировать"\ , выписав последнюю строку:
\[
f(0, 0, ..., 0, 0), f(0, 0, ..., 0, 1), ..., f(1, 1, ..., 1, 1)
\]
Такой способ задания булевой функции называется задание \underline{вектором значений}.

\parag{4. Существенные и фиктивные переменные булевой функции.}
Переменная $x_{i}$ называется \textit{существенной} переменной функции булевой функции f, если существует такой набор значений $a_{1}, ..., a_{i-1}, a_{i + 1}, ..., a_{n}$, что:
\[
f(a_{1}, ..., a_{i-1}, 0, a_{i+1}, ..., a_{n}) \neq f(a_{1}, ..., a_{i-1}, 1, a_{i+1}, ..., a_{n})
\]
В противном случае переменная $x_{i}$ называется \textit{фиктивной}.


\parag{5. Дизъюнктивная нормальная форма.}
Дизъюнктивная нормальная форма (ДНФ) - это представление булевой формулы в виде дизъюнкции конъюнктов литералов. Любая булева формула может быть приведена к ДНФ. \textit{Литерал} - это x или $\overline{x}$, где $x$ - некая логическая переменная. \textit{Конъюнкт} - это конъюнкция литералов. Например, $x1 \land x2 \land x3$ - конъюнкт. k-дизъюнктивной нормальной формой называют ДНФ, в которой каждая конъюнкция содержит ровно k литералов.

\parag{6. Множество, подмножество, равенство множеств.}
\textit{Множество} — это совокупность каких-то элементов, полностью определяемая своими элементами. Элементами множества могут быть другие множества.

\noindent
Будем говорить, что элемент $x$ \textit{принадлежит} множеству A, если он является его элементом. Обозначение: $x \in A$ (эта запись означает утверждение и принимает логические значения "истина"\ , "ложь"\ – входит или не входит в множество). 

\noindent
Если любой элемент множества A принадлежит множеству B, то множество A называется \textit{подмножеством} множества B, обозначение $A \subseteq B$.

\noindent
\textit{Равенство множеств} A = B — это утверждение, которое означает, что множества состоят из одних и тех же элементов. То есть: любой элемент множества A принадлежит множеству B и любой элемент множества B принадлежит множеству A.

\noindent
Есть уникальное множество - \textit{пустое}, - которое не содержит никаких элементов. Обозначение: $\varnothing$. 

\noindent
Если элементов в множестве мало, его можно задать, указав все эти элементы (в фигурные скобки). Порядок не играет роли. Поэтому \{a, b, c, d\} = \{d, a, c, b\}. 

\noindent
Количество элементов в множестве A, если оно конечно, непустое, обозначается |A| и называется \textit{мощностью множества}.


\parag{7. Операции с множествами: объеднинение, пересечение, разность, симметрическая разность. Диаграммы Эйлера-Венна.}
1) \textit{Объединение множеств}. Обозначение $A \cup B$. Это множество, состоящее в точности из тех элементов, которые принадлежат хотя бы одному из множеств A и B.

\noindent
2) \textit{Пересечение множеств}. Обозначение $A \cap B$. Это множество, состоящее в точности из тех элементов, которые принадлежат обоим множествам A и B.

\noindent
3) \textit{Разность множеств}. Обозначение $A \setminus B$. Это множество, состоящее в точности
из тех элементов, которые принадлежат множеству A, но не принадлежат множеству B.

\noindent
4) \textit{Симметрическая разность множеств}. Обозначение $A \bigtriangleup B$. Это множество, состоящее в точности из тех элементов, которые принадлежат ровно одному из множеств:
либо A, либо B.

\noindent
5) \textit{Диаграммы Эйлера-Венна} - геометрическая схема, с помощью которой можно изобразить отношения между множествами, для наглядного представления. При этом способе множество изображается условным кругом (или
другой геометрической фигурой) и предполагается, что внутренность круга изображает элементы множества. (Я думаю, нарисовать сможете).

\parag{8. Законы Моргана (с обобщением на произвольное семейство множеств).}
Законы де Моргана задают правило взятия отрицания от конъюнкции и дизъюнкции:
\[ \neg(x \vee y) = \neg x \wedge \neg y \]
\[ \neg(x \wedge y) = \neg x \vee \neg y \]
(Доказывается по таблицам истинности).

\noindent
Равенства обобщаются на случай нескольких переменных:
\[ \neg(x_{1} \vee x_{2} \vee ... \vee x_{n}) = \neg x_{1} \wedge \neg x_{2} \wedge ... \wedge \neg x_{n} \]
\[ \neg(x_{1} \wedge x_{2} \wedge ... \wedge x_{n}) = \neg x_{1} \vee \neg x_{2} \vee ... \vee \neg x_{n} \]

\noindent
Доказательство (для первой формулы, анналогично для второй):

\begin{enumerate}
    \item База: Выражение верно для n = 2: $\neg(x_{1}  \vee x_{2}) = \neg x_{1} \wedge \neg x_{2}$
    \item Предположение: Пусть верно для n = k - 1: \[ \neg(x_{1} \vee x_{2} \vee ... \vee x_{k - 1}) = \neg x_{1} \wedge \neg x_{2} \wedge ... \wedge \neg x_{k - 1} \]
    \item Шаг: проверим для n = k. Сделаем замену $y = x_{1} \vee x_{2} \vee ... \vee x_{k - 1}$
        $$ \neg(x_{1} \vee x_{2} \vee ... \vee x_{k}) = \neg(y \vee x_{k}) = \neg y \wedge \neg x_{k} = \neg (x_{1} \vee x_{2} \vee ... \vee x_{k - 1}) \wedge \neg x_{k} = $$
        $$= \neg x_{1} \wedge \neg x_{2} \wedge ... \wedge \neg x_{k - 1} \wedge \neg x_{k}$$
        
        (Последний переход выполнен по предположению индукции)
\end{enumerate}  

\noindent
(*) На языке множеств законы де Моргана формулируются так: (I) элемент x не
принадлежит объединению семейства множеств тогда и только тогда, когда он не
принадлежит ни одному из этих множеств; (II) элемент x не принадлежит пересечению семейства множеств тогда и только тогда, когда он не принадлежит хотя бы одному из этих множеств. В таком виде законы де Моргана применимы и к
бесконечным семействам множеств.

\parag{9. Закон контрапозиции.}
\textit{Принцип контрапозиции} - теорема равносильна обратной к противоположной. Тождество $$x \to y = \neg y \to \neg x$$
выражает принцип контрапозиции. Этот принцип часто используется в математических доказательствах: вместо
доказательства утверждения «если А, то Б» зачастую удобнее изменить посылку
и доказывать равносильное утверждение «если не Б, то не А». Проверка тождества легко производится по таблице истинности.

\parag{10. Правило суммы.}
Правило суммы для множеств. Если какое-то множество A разделено на две части B
и C, не имеющие общих элементов, то |A| = |B| + |C|.
(определение из чернивика книги. не примут - тыкните их в собственную книгу)

\noindent
Комбинаторное правило суммы. Пусть объект A можно выбрать M способами, а объект B можно выбрать N способами, причём выбор одного объекта исключает одновременный выбор другого объекта. Тогда выбрать A или B можно M + N способами.

\parag{11. Метод математической индукции.}
Доказательства по индукции применяются, когда есть последовательность утверждений $$ A_{1}, A_{2}, A_{3}, ..., A_{n}, ... $$ и мы хотим доказать, что все они верны. Принцип индукции говорит, что для этого
достаточно сделать две вещи:
\begin{enumerate}
    \item Базис индукции: надо доказать, что $A_{1}$ (первое утверждение в цепочке) верно.
    \item Шаг индукции: надо доказать (для произвольного n), что $A_{n+1}$ верно, предполагая известным, что $A_{n}$ верно.
\end{enumerate}

\noindent
Мы должны доказать, что из $A_{n}$ следует $A_{n+1}$. Доказав следование и базу, мы можем применить шаг индукции к $A_{1}$ и получить $A_{2}$. Постепенно применяя шаг, мы дойдем до любого $A_{n}$.

\parag{12. Формула включений и исключений.}
Формула включений-исключений обобщает правило суммы и даёт выражение для объединения нескольких, возможно пересекающихся, множеств.

Пример для двух множеств:
\[  |A \cup B| = |A| + |B| - |A \cap B| \]

Пример для трех множеств:
\[ |A \cup B \cup C| = |A| + |B| + |C| - |A \cap B| - |A \cap C| - |B \cap C| + |A \cap B \cap C| \]

Общий вид:
\[ |A_{1} \cup A_{2} \cup ... \cup A_{n}| = |A_{1}| + ... + |A_{n}| - |A_{1} \cap A_{2}| - |A_{1} \cap A_{3}| - ... + |A_{1} \cap A_{2} \cap A_{3}|+ \]
\[  + |A_{1} \cap A_{2} \cap A_{4}| + ... - (-1)^{n}|A_{1} \cap A_{2} \cap ... \cap A_{n}| \]

\noindent
Удобно представить итоговую формулу в более компактном виде. Для этого введём обозначения. Через S будем обозначать подмножество множества \{1, ..., n\}, каждое такое подмножество выделяет некоторое семейство подмножеств $$\{A_{i}
: i \in S\}$$

\noindent
Через $A_{S}$ обозначим пересечение всех множеств, входящих в семейство S, т.е.
$$ A_{S} = \underset{i \in S}{\cap}A_{i} $$

\noindent
В таких обозначениях формула включений-исключений записывается достаточно компактно:
$$ |A_{1} \cup A_{2} \cup ... \cup A_{n}| = \underset{S \neq \varnothing}{\sum} (-1)^{|S|+1}|A_{S}| $$

\parag{13. Правило произведения.}
\textit{Из лекций:} Если есть N способов выбрать 1-ый объект и после каждого выбора есть M способов выбрать 2-ой объект, то всего есть $N \times M$ способов выбрать два объекта.

\noindent
\textit{Из книги:} Правило произведения. Если объект интересующего нас вида строится в несколько шагов (1, 2, ..., k), и на каждом шаге есть выбор из какого-то числа вариантов ($m_{1}, m_{2}, . . . , m_{k}$), причём количество выборов на каждом шаге не зависит от сделанных ранее выборов, то общее количество объектов N равно произведению количеств вариантов выбора для
каждого из шагов: $N = m_{1} \times m_{2} \times ... \times m_{k}$.

\parag{14. Комбинаторные числа. Число перестановок, число подмножеств размера k у n-элементного множества}
Пусть $A = \{a_{1}, ..., a_{n}\}$ - множество из n элементов. 

\noindent
\textit{Комбинаторный объект} - это подмножество с определенными свойствами из элементов множества A. 

\noindent
\textit{Комбинаторное число} (связанное с комбинаторным объектом) - это количество комбинаторных объектов этого вида.

\noindent
Некоторые комбинаторные числа имеют собственные названия и устоявшиеся обозначения.

\vskip 0.4em

\noindent
В комбинаторике \textit{размещением из n по k} называется упорядоченный набор из k различных элементов из некоторого множества различных k элементов. (1,3,2,5) — это 4-элементное размещение из 6-элементного множества $\{1,2,3,4,5,6\}$

\[
    A^{k}_{n} = n(n-1)(n-2)...(n-k+1) = \frac{n!}{(n - k)!}
\]

\noindent
\textit{Перестановка} — это упорядоченный набор из чисел 1, 2, ..., n, в котором числу i сопоставляется i-ый элемент из набора. Другими словами, это биекция на множестве $\{1,2, ... ,n\}$. Например, (2, 1, 3) — это перестановка (1, 2, 3). Число всех перестановок обозначают за $P_{n}$. Так как перестановка — это то же самое, что размещение по n элементам, то
\[
    P_{n} = A_{n}^{n} = \frac{n!}{(n -n)!} = \frac{n!}{0!} = n!
\]

\noindent
\textit{Сочетаниями из n по k} называется набор k элементов, выбранных из данного множества, содержащего n различных элементов. Наборы, отличающиеся только порядком следования элементов (но не составом), считаются одинаковыми, этим сочетания отличаются от размещений.
\[
    C^{k}_{n} = \frac{A_{n}^{k}}{k!} = \frac{n!}{(n - k)!k!} = {n \choose k}
\]

\noindent
Сочетанием с повторениями называются сочетания, в которых каждый элемент набора может встречаться несколько раз. Количество сочетаний с повторениями из n по k равно $C^{k}_{n+k-1} = C^{n-1}_{n+K-1}$. (можно доказать с помощью метода точек и перегородок). 

\parag{15. Характеристическая функция и её использование при подсчёте числа элементов множества.}
Характеристическая функция - это функция, определённая на множестве X, которая указывает на принадлежность элемента x, принадлежащего X, подмножеству A.

\begin{equation*}
 \begin{matrix}
    \chi_{A}(x) = 
 \end{matrix}
 \begin{cases}
    1, x \in A \\
    0, x \notin A
 \end{cases}
\end{equation*}

\noindent
Можно определить понятие мощности подмножества A на множестве X, используя характеристическую функцию:
\[
    |A| = \underset{x \in X}{\sum} \chi_{A}(x)
\]

\parag{16. Функции. Область определения и множество значений.}
\textit{Функцией} из множества A в множество B мы назовём такое соответствие, которое сопоставляет некоторым элементам множества A какой-то элемент множества
B. Данному $x \in A$ (его называют аргументом функции) функция f из A в B либо
не сопоставляет никакого элемента в B, либо сопоставляет ровно один такой элемент y.

\noindent
\textit{Область определения} функции f из A в B состоит в точности из тех элементов $x$ множества A, которым сопоставлен элемент f(x) множества B.

Обозначение: $Dom(f) = \{a | a \in A; \exists b: f(a) = b\}$

\noindent
Элементы $f(x)$ для всех $x$ из области определения функции f образуют \textit{множество значений} функции f.

Обозначение: $Range(f) = \{b | \exists a \in A: f(a) = b\}$

\parag{17. Биномиальные коэффициенты, основные свойства. Бином Ньютона.}
Хорошо известны формулы раскрытия скобок (формулы сокращенного умножения). Например, $(a + b)^{2} = a^2 + 2ab + b^2$ . Оказывается, что есть подобная формула для любой целой неотрицательной степени — \textit{бином Ньютона}:
\[
    (a+b)^{n} = {n \choose 0}a^n + {n \choose 1}a^{n-1}b + ... + {n \choose k}a^{n-k}b^k + ... + {n \choose n}b^n = \overset{n}{\underset{k=0}{\sum}}{n \choose k}a^{n-k}b^k
\]
$$ где {n \choose k} - \textit{биномальный коэфиициент.}$$

\noindent
Свойства биномальных коэффициентов:
\begin{itemize}
    \item ${n \choose k} = {n - 1 \choose k} + {n - 1 \choose k - 1}$
    \item k-й коэффициент есть количество сочетаний из n по k - 1.
    \item ${n \choose k} = \frac{n!}{(n-k)!k!}$
\end{itemize}

\parag{18. Треугольник Паскаля. Рекуррентное соотношение}
Треугольник Паскаля - бесконечная треугольная таблица, в которой на вершине и по боковым сторонам стоят единицы, каждое из остальных чисел равно сумме двух чисел, стоящих над ним в предшествующей строке.

\vspace{30px}
\includegraphics[width=\textwidth]{images/pascal_traingle.png}
\vspace{30px}

Пусть T (n, k) - k-й элемент n-й строки. Свойства:
\begin{itemize}
    \item $T(n, k) = {n \choose k}$
    \item Симметричность строк: T(n, k) = T(n, n - k), n - строка, k - столбец.
    \item Возрастание чисел в первой половине строки: $T(n, i) < T (n, i + 1), 0 \leqslant i \leqslant n/2, i \in Z$
    \item $\sum_{k=0}^{n} T(n, k) = 2^n$
    \item $\sum_{k=0}^{n} (-1)^k T(n, k) = 0$
\end{itemize}

\noindent
\textit{Рекуррентная соотношение} - последовательность, в которой каждый следующий член выражается через предыдущие элементы и, возможно, номер элемента.

\noindent
\textit{Рекуррентная формула} - формула вида $a_{n}=f(n,a_{n-1},a_{n-2},\dots ,a_{n-p})$, выражающая каждый член последовательности $a_{n}$ через p предыдущих членов и возможно номер члена последовательности n.

\parag{19. Образы и прообразы множеств. Полный прообраз.}
Пусть функция f из множества A в множество B устанавливает соответствие между элементами множеств A и B. Пусть $X \subseteq A$ – подмножество множества A. Функция f сопоставляет ему \textit{образ} $f(X) \subseteq B$ подмножества X. По определению f(X) состоит в точности из тех
элементов множества B, которые являются значениями элементов из X. Используя введённые нами для множеств обозначения, это можно записать как
\[
    f(X) = \{ b | \exists x \in X: b = f(x) \}
\]
Совокупность всех тех элементов $a \in A$, образом которых является данный элемент $b = f(a)$, $f(a) \in B$ , называется \textit{прообразом} элемента $b$ и обозначается $f^{-1}(b)$

\noindent
Подмножеству $Y \subseteq B$ можно сопоставить \textit{полный прообраз} $f^{-1}(Y) \subseteq A $ подмножества Y. По определению $f^{-1}(Y)$ состоит в точности из тех элементов A, значения которых лежат в Y. Или формально:
\[
    f^{-1}(Y) = \{a | f(a) \in Y\}
\]

\parag{20. Отображения (всюду определённые функции). Инъекции, сюръекции и биекции.}
Отображение множества A в множество B - функция, которому \underline{каждому} элементу $a \in A$ ставит в соответствие элемент $b \in B$.

\noindent
Пусть $f: A \to B$. Тогда функция f называется:

1) \textit{Инъективной (или инъекцией)}, если:
\[
    (b = f(a_{1}))\ \&\ (b = f(a_{2})) \Rightarrow (a_{1} = a_{2})
\]

2) \textit{Сюрьективной (или сюрьекцией)}, если:
\[
    \forall b \in B\ \exists a \in A: b = f(a)
\]

3) \textit{Биективной (или биекцией)}, если \textit{она сюрьективна и инъективна}. Биекции также называют \textit{взаимно-однозначными функциями}. 

\parag{21. Бинарные отношения. Транзитивность, симметричность, рефлексивность.}
\textit{Бинарным отношением между множествами A и B} называется подмножество R произведения $A \times B$. В том случае, когда A = B, мы говорим просто об отношении R на A.

Для бинарных отношений часто используется инфиксная форма записи:
\[
    aRb \overset{Def}{=} (a, b) \in R \subset A \times B 
\]

Пусть $R \subset A^{2}$. Тогда отношение R называется:

\begin{center}
\begin{tabular}{cc}
    \textit{рефлексивным}, & если $\forall a \in A\ (aRa)$ \\
    \textit{антирефликсивным}, & если $\forall a \in A\ \neg(aRa)$ \\
    \textit{симметричным}, & если $\forall a, b \in A\ (aRb \Rightarrow bRa)$ \\
    \textit{антисимметричным}, & если $\forall a, b \in A\ (aRb\ \&\ bRa \Rightarrow a = b)$ \\
    \textit{транзитивным}, & если $\forall a, b, c \in A\ (aRb\ \&\ bRc \Rightarrow aRc)$ \\
    \textit{линейным}, & если $\forall a, b \in A\ (a = b \lor aRb \lor bRa)$
\end{tabular}
\end{center}

\parag{22. Теоретико-множественные операции с отношениями. Операция обращения.}
Бинарные отношения — это множества пар элементов, связанных этими отношениями, поэтому к отношениям применимы все операции, выполняемые над множествами. Пусть $P \subseteq A \times A$ и $Q \subseteq A \times A$, тогда:

1) Пересечение отношений $P \cap Q$ - отношение, которое содержит только те упорядоченные пары, которые есть и в P и в Q:
\[
    P \cap Q = \{(x, y)\ |\ (x, y) \in P \land (x, y) \in Q\}
\]

2) Объединение отношений $P \cup Q$ - отношение, которое содержит все упорядоченные пары отношения P и все упорядоченные пары отношения Q:
\[
    P \cup Q = \{(x, y)\ |\ (x, y) \in P \lor (x, y) \in Q\}
\]

3) Разность отношений $P \setminus Q$ - отношение, которое содержит только те упорядоченные пары, которые содержатся в P, но не содержатся в Q:
\[
    P \setminus Q = \{(x, y)\ |\ (x, y) \in P \land (x, y) \notin Q\}
\]

4) Симметрическая разность отношений $P \bigtriangleup Q$ - отношение, которое содержит только те упорядоченные пары, которые содержатся в объединении P и Q, но не содержатся в пересечении P и Q:
\[
    P \bigtriangleup Q = \{(x, y)\ |\ (x, y) \in (P \cup Q) \land (x, y) \notin (P \cap Q)\}
\]

5) Дополнение отношения P - это отношение, состоящее из всех пар $(x, y) \in (A \times A)$, которые не входят в отношение P:
\[
    \bar P = \{(x, y)\ |\ (x, y) \in (A \times A) \land (x, y) \notin P\}
\]

6) Обратным отношением $P^{-1}$ к P называется такое отношение, которое содержит пару (x, y) тогда и только тогда, когда P содержит пару (y, x):
\[
    P^{-1} = \{(x, y)\ |\ (y, x) \in P\}
\]

\parag{23. Композиция бинарных отношений}
Пусть $R_{1} \subset A \times C$ - отношение между A и С, а $R_{2} \subset C \times B$ - отношение между C и B. \textit{Композицией} двух отношений $R_{1}$ и $R_{2}$ называется отношение $R \subset A \times B$ между A и B, определяемое следующим образом:
\[
    R \overset{Def}{=} R_{1} \circ R_{2} \overset{Def}{=} \{(a, b)\ |\ a \in A\ \&\ b \in B\ \&\ \exists c \in C\ (aR_{1}c\ \&\ cR_{2}b)\}
\]
Другими словами, $aR_{1} \circ R_{2}b \iff \exists c \in C\ (aR_{1}c\ \&\ cR_{2}b)$.

\parag{24. Отношения эквивалентности.}
Отношение на некотором множестве, которое одновременно рефлексивно, симметрично и транзитивно, называют \textit{отношением эквивалентности}.

\noindent
\textit{Классы эквивалентности} - непересекающиеся подмножества множества X, при этом любые два элемента одного класса находятся в
отношении R, а любые два элемента разных классов не находятся в отношении R.

\parag{25. Графы. Основные определения: ребра, вершины, степени вершин.}
Граф $G$ - совокупность двух множеств - непустого множества V (множества вершин) и множества E двухэлементных подмножеств множества V (E - множество ребер).
\[
    E = \{ \{u, v\}\ |\ u, v \in V, u \neq v \}
\]
\textit{Степень вершины $u$} - количество вершин, смежных с $u$. Обозначение: $d(u)$.

\parag{26. Пути и циклы в графах.}
\textit{Путь} - последовательность вершин $v_{1}, v_{2}, ..., v_{n}$, такая что $\forall i \in \{1, 2, ..., n - 1\}$: $v_{i}$ и $v_{i+1}$ - соединены ребром.


\noindent
\textit{Простой путь} - путь, такой что в нем все вершины различны.

\noindent
\textit{Цикл} - путь, такой что $v_{1} = v_{2}$

\noindent
\textit{Простой цикл} - цикл, все вершины которого, кроме первой и последней различны.

\parag{27. Отношение достижимости (связанности) и компоненты связности графа.}
Говорят, что вершина $v$ достижима из вершины $u$, если существует путь $v_{1}, v_{2}, ..., v_{n}$, где $v_{1} = u$, а $v_{n} = v$.
Связность вершины $u$ и $v$ означает их достижимость друг из друга.

\noindent
Обозначение связности: $u \rightarrow v$

\noindent
Основные свойства:

1) $u \rightarrow u\ \forall u \in V$

2) если $u \rightarrow v$, то $v \rightarrow u$ (для неор.графов)

3) Транзитивность (для неор.графов): если  $u \rightarrow v$ и  $v \rightarrow w$, то  $u \rightarrow w$

\noindent
\textit{Компонента связности} - подмножество множества вершин V графа G такое, что любая пара вершин в этом подмножестве связаны, а также любая вершина этого подмножества не связана с любой другой не из этого подмножества.

\noindent
\textit{Компонента связности} - класс эквивалентности по отношению достижимости (валидно, т.к. отношение достижимости является отношением эквивалентности, это следует из свойств).

\parag{28. Правильные раскраски графов. Формулировка критерия 2 - раскрашиваемости}
Раскраска вершин графа называется правильной, если концы каждого ребра покрашены в разные цвета. \textit{2-раскрашиваемые графы} - это графы, которые можно правильно раскрасить в 2 цвета. Граф 2-раскрашиваемый, когда в нем нет циклов нечетной длины. 

\noindent
Теорема. 2 - раскраска описанного типа возможна тогда и только тогда, когда в графе нет циклов нечётной длины.

\parag{29. Двудольные графы. Двудольные и двураскрашиваемые графы.}
\textit{2-раскрашиваемые графы} - это графы, которые можно правильно раскрасить в 2 цвета.

\noindent
Граф G(V, E) называется \textit{двудольным}, если множество V может быть разбито на два непересекающихся множества $V_{1}$ и $V_{2}$ ($V_{1} \cup V_{2} = V, V_{1} \cap V_{2} = \varnothing$), причем всякое ребро из E инцидентно вершине из $V_{1}$ и вершине из $V_{2}$ (соединяет вершину из $V_{1}$ с вершиной из $V_{2}$). Множества $V_{1}$ и $V_{2}$ называются \textit{долями} двудольного графа.

\parag{30. Подграфы. Изоморфизм графов. Клики и независимые множества.}
Граф $G'(V', E')$ называется \textit{подграфом} графа G(V, E), если $V' \subset V\ \&\ E' \subset E$.

\noindent
\textit{Изоморфизм графов.}
Говорят, что два графа, $G_{1}(V_{1}, E_{1})$ и $G_{2}(V_{2}, E_{2})$, \textit{изоморфны}, если существует изоморфизм $h: V_{1} \rightarrow V_{2}$, такой что две вершины $u$ и $v$ графа $G_{1}$ смежны тогда и только тогда, когда вершины $h(u)$ и $h(v)$ смежны в графе $G_{2}$.


\noindent
\textit{Кликой} называется такое подмножество вершин графа, каждая пара которых связана ребром.

\noindent
\textit{Независимым множеством} называется такое подмножество вершин графа, никакая пара которых не связана ребром.


\parag{31. Эйлеровы циклы.}
Цикл называется \textit{эйлеровым}, если он проходит по всем рёбрам графа, причём только один раз.

\noindent
Критерии существования. Эйлеров цикл существует в неориентированном графе тогда и только тогда когда выполнены условия:

1) Граф связен.

2) Все вершины имеют четную степень.

3) Цикл проходит через все ребра.

\noindent
Эйлеров цикл существует в ориентированном графе тогда и только тогда когда выполнены условия:

1) Граф сильно связен.

2) Для любой вершины v верно $d^{-}(v) = d^{+}(v)$


\parag{32. Деревья. Полные бинарные деревья.}
\textit{Дерево} - связный граф без простых циклов длины $\geqslant$ 3. Для деревьев  выполняются эти свойства:

1) G - связный граф, где нельзя удалить ни одного ребра без нарушения связности.

2) G - связный граф, где число рёбер на единицу меньше числа вершин.

3) G - связный граф, где для любых двух вершин u, v существует единственный простой путь из u в v.

4) G - cвязный граф, где нет простых циклов длины больше 2.

(Все 4 определения эквиваленты)

\vskip 0.2in

\noindent
Еще свойства деревьев:

1) Для любых трёх вершин дерева, пути между парами этих вершин имеют ровно одну общую вершину. 

2) Из любого связного неориентированном графа можно получить дерево удалением части ребер. (Полученное дерево будет называться остовным деревом исходного графа). 

3) В любом дереве есть висячая вершина.

\vskip 0.2in

\noindent
\textit{Полное бинарное дерево глубины n} — неориентированный граф-дерево, вершины которого — двоичные слова длины не больше n, а рёбра соединяют слова, которые можно получить друг из друга добавлением (или исключением соответственно) символа в конец слова. Корень дерева — пустое слово.


\parag{33. Ориентированные графы, основные определения.}
\textit{Ориентированный граф (орграф)} - такой граф, в котором направление ребра имеет значение; ребра (v, u) и (u, v) - разные. Записать можно следующим образом: $E = \{(A, B), (C, D), ...\}$. Часто ребра в орграфе называют \textit{дугами}. 

\noindent
Для орграфов и неорграфов верно, что $u \rightarrow v$, $v \rightarrow w \Rightarrow u \rightarrow w$. 

\noindent
Для ориентированных графов отличают входящую степень $d^{-}(v)$ — количество входящих в вершину ребер, то есть ребер вида (u, v), и исходящую степень $d^{+}(v)$ — количество исходящих ребер, то есть ребер вида (v, u).

\noindent
Путь, простой путь, цикл, простой цикл - эквивалентно определениям для неорграфа.

\parag{34. Компоненты сильной связности ориентированного графа.}
Будем говорить, что вершины u и v сильно связны, когда $u \rightarrow v$ и $v \rightarrow u$. Обозначается как $u \leftrightarrow v$.

\noindent
\textit{Компонента сильной связности} — определено только для орграфов; аналогично обычной компоненте связности, но между любой парой вершин в компоненте должна быть сильная связность. 

\noindent
Теорема: Вершины ориентированного графа можно однозначно разбить на непересекающиеся группы, называемые сильно связными компонентами, при этом:

1) Каждая вершина графа попадает ровно в одну группу;

2) Любые две вершины из одной группы сильно связаны;

3) Любые две вершины из двух разных групп не являются сильно связанными (в одну из сторон — или даже в обе — пути нет).

\parag{35. Отношения частичного порядка (строгие и нестрогие), линейные порядки}
\textit{Отношение порядка} - антисимметричное транзитивное отношение.

\noindent
\textit{Нестрогое отношение порядка} - рефлексивное отношение порядка.

\noindent
\textit{Строгое отношение порядка} - антирефлексивное отношение порядка.

\noindent
\textit{Линейное отношение порядка} - отношение порядка, обладающее свойством линейности ($\forall a, b \in A\ (a = b \lor aRb \lor bRa)$).

\noindent
\textit{Частичное отношение порядка} - отношение порядка, не обладающее свойством линейности.

\vskip 0.2 em

Обычно отношение строгого порядка (линейного или частичного) обозначается знаком $\prec$, а отношение нестрогого порядка - знаком $\preceq$.

\parag{36. Отношение непосредственного следования}
\textit{Отношение непосредственного следования} - это отношение порядка $\prec$, такое что если $x \prec y$, то $\nexists z: x \prec z \land z \prec y$. Кратко записывается так:
\[
    \prec = \{(x, y)\ |\ (x \prec y) \land \neg(\exists z: x \prec z \land z \prec y)\}
\]

\parag{37. Изоморфные отношения частичного порядка}
Отношения частичного порядка $R_{1} \subseteq A \times A$ и $R_{2} \subseteq B \times B$ называются изоморфными, если существует такая биекция $f : A \rightarrow B$, что для $\forall x, y$ $xR_{1}y \iff f(x)R_{2}f(y)$

\end{document}
